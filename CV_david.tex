%Document Class Definition
\documentclass{article}

%Required Packages
\usepackage{xcolor}
\usepackage[utf8]{inputenc}
\usepackage[french]{babel}
\usepackage{amsfonts}
\usepackage{amssymb}
\usepackage[T1]{fontenc}      % LaTeX
\usepackage{fontawesome}
\usepackage{tikz}
	\usetikzlibrary{babel} %correction for the french version of the document
	\usetikzlibrary{shapes.geometric}
	\usetikzlibrary{calc}
	\usetikzlibrary{decorations.pathreplacing}
\usepackage[left=1cm,right=1cm,top=1cm,bottom=1cm]{geometry}
\usepackage{setspace}
\usepackage[absolute,overlay]{textpos}
\usepackage{hyperref}
\usepackage{tabularx}
\usepackage{calc}
\usepackage{ifthen}
\usepackage{coolstr}
\usepackage{textcase}
\usepackage{amsmath}
\usepackage{relsize}
\usepackage{moresize}
\usepackage{times}
\usepackage{enumitem}
\usepackage{booktabs}
\usepackage{multirow}
\usepackage{colortbl}
\usepackage{mathabx}

%Definition of the colors
\definecolor{white}{RGB}{255,255,255}
\definecolor{sidecolor}{RGB}{200,200,200}
\definecolor{mainblue}{RGB}{55,55,135}
\definecolor{maingray}{RGB}{144,144,144}

%Define CV Variables
\newcommand{\cvfirstname}{David}
\newcommand{\cvlastname}{Beauchemin}
\newcommand{\cvtitle}{B. Sc. Actuariat}

\newcommand{\cvemail}{david.beauchemin.5@ulaval.ca}
\newcommand{\cvphonenumber}{(514) 250-3616}
\newcommand{\cvaddress}{2435, Chemin Sainte-foy, app 1B\newline Québec}
\newcommand{\cvlocation}{Québec, Canada}
\newcommand{\cvlinkedin}{https://www.linkedin.com/in/david-beauchemin-918343108/}
\newcommand{\cvwhitewords}{Agile,Applications,ASP.NET,Build,C\#,Code,Collaborate,Data,Data Scientist,
Design,Development,Engineering,Environment,Java,Javascript,Knowledge,Machine Learning,Mobile,.NET,Product,
Projects,Software,Solutions,SQL,Team,Technologies,Testing,Tools,Web}

%Command for vertical centering of text into tabular
\renewcommand{\tabularxcolumn}[1]{m{#1}}

%Command for horizontal centering of specified width into tabular
\newcolumntype{M}[1]{>{\centering\arraybackslash}m{#1}}

%Command for bold font and textcolor to apply to a specific column into tabular
\newcolumntype{A}[1]{>{\bfseries\centering\color{mainblue}}m{#1}}

%Command to print checkmark with desired format into Certificate Table
\newcommand{\CheckMark}{\textcolor{mainblue}{\faCheck}}

%Command to create the correct seperator for Interests
\newcommand{\separator}{\textcolor{mainblue}{ $\blackdiamond$ }}

%Command to produce a round shaped corner framebox with defined colors
\newcommand{\mybox}[1]{
	\tikz
	\node[rectangle,rounded corners=1mm,draw=mainblue,anchor=text,fill=mainblue,inner sep=3pt,minimum height=0.5cm]
		{\small\bfseries\textcolor{white}{#1}};}

%Command to create the sidebar layout
\newcommand{\sidebar}{
	\begin{tikzpicture}[overlay]
		\node [rectangle, fill=sidecolor, anchor=north west, minimum width=9.25cm, 
		minimum height=\paperheight+2cm] (box) at (-2cm,3cm){};%
	\end{tikzpicture}}

%Command for printing the margin section titles
\newcommand{\profilesection}[1]{
	\vspace{6pt}
	{\bfseries\Large #1\;\rule[0.15\baselineskip]{7cm-\widthof{\bfseries\Large #1\;}}{1pt}\normalfont\normalsize}
	\vspace{6pt}}

%Command for printing the icons
\newcommand{\icon}[1]{
	\begin{tikzpicture}[baseline=(char.base)]
		\node[circle,draw,baseline=(char.base),mainblue,fill=mainblue,text=white,
		inner sep=0.25pt,minimum size=10mm,align=center,text centered](char){\large\textbf{{#1}}};
	\end{tikzpicture}}

%Command to create the star scale
\newcommand\starscale[2]{
	\pgfmathsetmacro\pgfxa{#1+1}
	\tikzstyle{scorestars}=[star,star points=5,star point ratio=2,thick,mainblue,scale=1.5,draw,inner sep=0.125em,anchor=outer 	point 3]
	\begin{tikzpicture}[baseline]
		\foreach \i in {1,...,#2} {
			\pgfmathparse{(\i<=#1?"mainblue":"maingray")}
			\edef\starcolor{\pgfmathresult}
			\draw (\i*1em,0) node[name=star\i,scorestars,fill=\starcolor]{};}
		\pgfmathparse{(#1>int(#1)?int(#1+1):0}
	\let\partstar=\pgfmathresult
	\ifnum\partstar>0
		\pgfmathsetmacro\starpart{#1-(int(#1))}
		\path [clip] ($(star\partstar.outer point 3)!(star\partstar.outer point 2)!(star\partstar.outer point 4)$) 
		rectangle 
		($(star\partstar.outer point 2 |- star\partstar.outer point 1)!\starpart!(star\partstar.outer point 1 -| star\partstar.outer point 5)$);
		\fill (\partstar*1em,0) node[scorestars,fill=mainblue]{};
	\fi
	\end{tikzpicture}}

%Command for printing skill progress bars
\newcommand{\progressbarthickness}{0.1}
\newcommand\skillbar[1]{
	\begin{tikzpicture}
		\foreach [count=\i] \x/\y in {#1}{
			\draw[fill=maingray,maingray] (0,\i) rectangle (7,\i+\progressbarthickness);
			\draw[fill=white,mainblue](0,\i) rectangle (\y,\i+\progressbarthickness);
			\node[circle,fill,mainblue] at (\y,\i+\progressbarthickness/2){};
			\node[circle,fill,maingray,scale=0.5] at (\y,\i+\progressbarthickness/2){};
			\node[above right] at (0,\i+\progressbarthickness){\x};}
	\end{tikzpicture}}

%Command for printing white words
\newcommand\printww[1]{
	\tiny\textcolor{white}{
	\foreach \x in #1{\x \- \,}}}

%Command to create section Title
\newcommand\sectiontitle[1]{
	\bfseries\LARGE
	\textcolor{mainblue}{\substr{#1}{1}{3}}%
	\textcolor{black}{\substr{#1}{4}{end}}
	\normalfont\normalsize}

%Command to let mark in order to trace tikz picture
\newcommand{\tikzmark}[1]{\tikz[overlay,remember picture] \node[circle,scale=0.01] (#1) {};}

%Command to insert a new Education Statement
\newcommand{\EduStatement}[5]{
	\normalfont\normalsize
	\vspace{-\baselineskip}
	\tikzmark{mark1}
	\vspace{12pt}\\
	#4\\
	\large{\textbf{#5}}\\
	\textcolor{mainblue}{#3\small\textbf{\hfill [#1 - #2]}}\\
	\normalfont\normalsize
	\tikzmark{mark2}
%	\begin{tikzpicture}[overlay, remember picture]
%		\draw [decoration={brace,mirror,raise=10pt,amplitude=1em},decorate,ultra thick,mainblue]
% 		(mark1) --  (mark2);
%	\end{tikzpicture}
	}

%Command to insert a new Experience Statement
\newenvironment{ExpStatement}[4]{
	\normalfont\normalsize
	\vspace{-\baselineskip}
	\tikzmark{mark1}
	\vspace{15pt}\\
	#3\\
	\large{\textbf{#4}}
	\small
	\begin{itemize}[label=\textcolor{mainblue}{\textbullet},topsep=0pt,partopsep=0pt,itemsep=0pt,parsep=0pt]}
		{\vspace{-\baselineskip}
	\end{itemize}
	\normalfont\normalsize
	\tikzmark{mark2}
%	\begin{tikzpicture}[overlay, remember picture]
%		\draw [decoration={brace,mirror,raise=10pt,amplitude=1em},decorate,ultra thick,mainblue]
% 		(mark1) --  (mark2);
%	\end{tikzpicture}
	\\}

\pagestyle{empty} % Disable headers and footers
\setlength{\parindent}{0pt} 
\setlength{\TPHorizModule}{1cm}
\setlength{\TPVertModule}{1cm}
\hypersetup{
colorlinks,
linkcolor={red!50!black},
citecolor={blue!50!black},
urlcolor={blue!80!black}}

%Edition of the content
\begin{document}
\sidebar
\begin{textblock}{7}(0.5,1)
	\HUGE
	\cvfirstname\\[6pt]
	\bfseries
	\cvlastname\\[6pt]
	\large
	\cvtitle\\
\end{textblock}

\begin{textblock}{7}(0.5, 3.75)
	\profilesection{Information générale}	
	\vspace{-\baselineskip}
	\begin{table}
		\begin{tabularx}{7cm}{cX}
			\icon{\faAt}&\bfseries\href{mailto:\cvemail}\cvemail\\
			\icon{\faPhone}&\cvphonenumber\\
			\icon{\faHome}&\cvaddress\\
			\icon{\faMapMarker}&\cvlocation\\
			\icon{\faLinkedin}&\href{\cvlinkedin}{Voir profil}\\
		\end{tabularx}
	\end{table}

	\profilesection{Compétences linguistiques}
	\vspace{-\baselineskip}
	\begin{table}
		\large
		\begin{tabularx}{7cm}{Xc}
			English & \starscale{3.8}{5}\\
			Français & \starscale{4.4}{5}\\
		\end{tabularx}
	\end{table}

	\profilesection{Langages de programmation}
	\skillbar{ Scilab /1.5, HTML/1,C/2,Git \faGithub/3.2,\LaTeX/5, Markdown /4.5, R/5.6,Python/5.5,VBA/5.5,SAS/2, SQL/1.5}
\end{textblock}

\begin{textblock}{11.35}(9.25, 0.25)
	\printww{\cvwhitewords}
\end{textblock}

\begin{textblock}{11.35}(9.25, 1)
	\sectiontitle{Formations}
	\EduStatement{09/2015}{Présent}{Université Laval}{Baccalauréat}{Actuariat}

	\sectiontitle{Expérience}
	\begin{ExpStatement}{[09/2016}{Actuel}{Université Laval - Actuel}{Auxiliaire de cours}
		\item Correction d'examens - Introduction à l'actuariat I
		\item Création d'exercices - Introduction à l'actuariat I
		\item Correction d'examens - Gestion du risque financier I
		\item Création d'un document d'exercice - Gestion du risque financier I
		\item Présentation sur RMarkdown et les outils de travail collaboratif - École d'actuariat
		\item Implication dans la création d'un package pour LaTeX - École d'actuariat
	\end{ExpStatement}
	\begin{ExpStatement}{11/2013}{10/2014}{Industrielle alliance, Assurance auto et habitation - [11/2013 - 10/2014] }{Agent en assurance de dommage}
		\item Compléter les soumissions clients
		\item Répondre aux besoins clients
		\item Conseiller les clients en assurance de dommage
		\item Supporter l'équipe des services financiers dans l'atteinte des objectifs
	\end{ExpStatement}
	\\
	\sectiontitle{Certifications}
	\vspace{3pt}
	\begin{table}
		\begin{tabularx}{\textwidth}{A{1.5cm}XM{1.5cm}}
			\toprule
			\centering\textcolor{black}{Examen}&\centering\textbf{Description}&\textbf{Date}\\
			\midrule
			P/1&Probability&01/2016\\
			FM/2&Financial Mathematics&06/2016\\
			ECON&VEE Economics&11/2016\\
			CORPFIN&VEE Corporate Finance&11/2016\\
			\bottomrule
 		\end{tabularx}
 	\end{table}
 	\vspace{\baselineskip}
	\sectiontitle{Aptitudes\ \&\ Forces}\\
	\mybox{Créativité}  \mybox{Leadership} \mybox{Curiosité intellectuelle} \mybox{Entrepreneur}\\
	\mybox{Travaillant} \mybox{Individualisation} \mybox{Esprit d'équipe} \mybox{Persévérance} \\
	\mybox{Sens des affaires} \mybox{Esprit critique} \mybox{Autodidacte} \mybox{Facilité d'apprentissage}\\
	\newline
	\sectiontitle{Accomplissements}\\
	Projets informatiques
	\begin{itemize}[label=\textcolor{mainblue}{\textbullet},topsep=0pt,partopsep=0pt,itemsep=0pt,
	parsep=0pt]
		\item Implication dans le développement d'un \href{http://ctan.org/pkg/actuarialsymbol}{package LaTeX} 
		\item Assemblage d'un serveur à domicile
		\item Assemblage et programmation d'un ordinateur portable Pi-Top
		\item Assemblage d'un centre média domestique
	\end{itemize}
	Sportif
	\begin{itemize}[label=\textcolor{mainblue}{\textbullet},topsep=0pt,partopsep=0pt,itemsep=0pt,
	parsep=0pt]
		\item Cross-country collégial 2010-2011
		\item Demi-Marathon : Montréal 2008, Québec 2012 (Meilleur temps 1h32)
		\item Marathon : Ottawa 2011, Québec 2011, Montréal 2011 (Meilleur temps 3h35)
		\item Triathlon demi-Ironman : Montréal 2011 (6h50)
	\end{itemize}
	\sectiontitle{Intérêts}\\
	Science \separator Lecture \separator Activités sportives \separator Hockey \separator Machine Learning \separator Voyage \separator Finance
\end{textblock}
\end{document}